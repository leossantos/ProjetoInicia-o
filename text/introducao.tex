\paragraph{}O Processamento de Linguagem Natural (PLN ou NLP, \textit{Natural Language Processing}) é um conjunto de técnicas para a análise automática e representação da linguagem natural~\cite{Cambria2014}, ou seja, a linguagem que é utilizada para comunicação entre os seres humanos, como por exemplo: Português, Inglês, Espanhol e assim por diante. Uma dessas técnicas consiste em um etiquetador sintático(\textit{POS-Tagger,Part of the Speech Tagger}), que trabalha com a classificação sintática de cada elemento em certo pedaço de texto. A classificação sintática automática tem uma importância prática muito relevante, principalmente em relação a potenciais aplicações, tais como: comando de fala, utilizado principalmente para \textit{Home Assistant} e celulares; reconhecimento de fala, hoje utilizado em celulares para desbloqueio de tela; correção ortográfica, utilizada em editores de texto, teclado de celulares, navegadores; e tradução automática ~\cite{Church1988}. 
\paragraph{}Entretanto a tarefa do \textit{POS-Tagger} não é tão simples, como por exemplo a palavra "cair" que por si só é um verbo, quando colocada em outro contexto como "cair da noite" se transforma em um substantivo num processo chamado nominalização~\cite{Rocha1999}. Logo, já é possível visualizar o desafio de um sistema automático de etiquetagem sintática, pois este deve levar em conta o contexto para determinar qual a função de um determinado vocábulo em uma frase.  Com base nisso este trabalho busca fazer uma análise comparativa das ferramentas existentes para o português do Brasil, levando em conta o desempenho de pelo menos 3 técnicas em um mesmo corpus de texto.
